% The contents of this file are subject to the terms of the Common Development
% and Distribution License (the License). You may not use this file except in
% compliance with the License.
%
%You can obtain a copy of the License at http://www.netbeans.org/cddl.html
% or http://www.netbeans.org/cddl.txt.
%
% When distributing Covered Code, include this CDDL Header Notice in each file
% and include the License file at http://www.netbeans.org/cddl.txt.
% If applicable, add the following below the CDDL Header, with the fields
% enclosed by brackets [] replaced by your own identifying information:
% "Portions Copyrighted [year] [name of copyright owner]"
%
%The Original Software is the LaTeX module.
%The Initial Developer of the Original Software is Jan Lahoda.
%Portions created by Jan Lahoda are Copyright 2002-2003.
%All Rights Reserved.
%
%Contributor(s): Jan Lahoda

\documentclass{report}
\usepackage{color}
\usepackage{graphicx}
\usepackage{html}
\usepackage{xspace}

\newcommand{\lenv}{\LaTeX{} authoring environment\xspace}
\newcommand{\done}{\textcolor[rgb]{0.0,1.0,0.0}{DONE}}

\title{\lenv}

\author{Jan Lahoda}

\begin{document}

\maketitle

\chapter{Programmer's Guide}

\section{Problems, Bugs and Future Improvements}

In this section ``TODOs'' are listed.

\begin{description}
\item[BUG]{\done{} Parser error recovery. Currently missing, causing parser to fail in case of
unfinished arguments.}
\item[BUG]{\done{} Code completion for environment arguments does not work (works like word completion).}
\item[BUG]{\done{} ``free'' text arguments are not supported}
\item[BUG]{\done{} fix the new \LaTeX{} file wizard (it does not generate the beginning \verb+\+).}
\item[BUG]{in the automata editor, if a command is used in the label, it is deleted.}
\item[FEATURE]{\verb+\include+ and \verb+\input+ code completion should provide
all files in the appropriate directory.}
\item[FEATURE]{more wizards, for example:
\begin{itemize}
\item{table/tabular (may also present a front-end to the borders --- nearly any combination
of borders is available but they are usually hard to imagine and code).}
\item{figure}
\end{itemize}
}
\end{description}

\subsection{Parser Error Recovery}

Currently, the parser provides basic parser recovery.

\begin{tabular}{| c | c | c |}
\hline
summary & description & status \\
\hline
\hline
closing argument at the end of the paragraph &
this is how \TeX{} does it, for all, except a few, commands when a \verb+\par+
(\verb+\n\n+) is found inside an argument, it simply inserts \} (or ]).
The commands which arguments may contain \verb+\par+ have to be marked. Proposed
tag is \verb+acceptspar+. &
PLANNED \\
\hline
\end{tabular}

\section{Command Hierarchy}

\section{Syntax Coloring}

\section{Command Attributes}

\begin{tabular}{| l | c | p{5cm} |}
\hline
attribute & used where & description \\
\hline
\hline
\verb+#argcountargument+ & & \\\hline
\verb+#caption+ & & \\\hline
\verb+#caption-command+ & & \\\hline
\verb+#captionable+ & & \\\hline
\verb+#cite+ & & \\\hline
\verb+#code+ & argument & the argument is code somehow, it should not be spell checked, etc.\\\hline
\verb+#documentclass+ & argument & the argument is the document class name\\\hline
\verb+#documentclassoptions+ & argument & the argument are the options for the given document class\\\hline
\verb+#enviroment-defining-command+ & & \\\hline
\verb+#environmentname+ & & \\\hline
\verb+#envname+ & & \\\hline
\verb+#figure-environment+ & & \\\hline
\verb+#include+ & & \\\hline
\verb+#label+ & & \\\hline
\verb+#no-parse+ & & \\\hline
\verb+#nonmandargvalueargument+ & & \\\hline
\verb+#package+ & argument & the argument is the package name\\\hline
\verb+#packageoptions+ & argument & the argument are the options for the given document class \\\hline
\verb+#ref+ & argument & the argument is the ref command argument\\\hline
\verb+#section-command+ & & \\\hline
\verb+#table-environment+ & & \\\hline
\verb+#test+ & & \\\hline
\verb+#text+ & & \\\hline
\verb+begin+ & command & the command is a begin like command\\\hline
\verb+end+ & command & the command is an end like command\\\hline
%advice_not
%class
%class-default
%default-encoding
%default-fontsize
%default-papersize
%fontsize
%icon_arrows
%icon_bigbrace
%icon_bigop
%icon_binop
%icon_brace
%icon_greek
%icon_symbols
%input
%label
%math
%package_amsfonts
%package_latexsym
%papersize
%par
%preamble
%symbols_arrows
%symbols_bigbrace
%symbols_bigop
%symbols_binop
%symbols_binrel
%symbols_brace
%symbols_greek
%symbols_symbols
%type
\end{tabular}

\chapter{Licenses}

The parts of the \lenv altogether with appropriate licence names are listed below:

\begin{table}
\begin{tabular}{| c | c |}
\hline
component & licence \\
\hline
NetBeans base & SPL \\
\hline
NetBeans external binaries & JavaHelp 2.0, ant, xerces \\
\hline
\LaTeX{} main modules & SPL \\
\hline
Viewer module & SPL\\
\hline
Multivalent & BSD? \\
\hline
dictionary (the ispell one) & BSD? \\
\hline
\LaTeX{} Help & ver.1.4 ??, ver. 1.6 GFDL \\
\hline
\LaTeX{} symbols & ???\\
\hline
\end{tabular}
\end{table}

\end{document}
